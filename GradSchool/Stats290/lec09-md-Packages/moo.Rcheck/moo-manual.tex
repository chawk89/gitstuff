\nonstopmode{}
\documentclass[a4paper]{book}
\usepackage[times,inconsolata,hyper]{Rd}
\usepackage{makeidx}
\usepackage[utf8,latin1]{inputenc}
% \usepackage{graphicx} % @USE GRAPHICX@
\makeindex{}
\begin{document}
\chapter*{}
\begin{center}
{\textbf{\huge Package `moo'}}
\par\bigskip{\large \today}
\end{center}
\begin{description}
\raggedright{}
\item[Type]\AsIs{Package}
\item[Title]\AsIs{What the Package Does Using Title Case}
\item[Version]\AsIs{1.0}
\item[Date]\AsIs{2020-01-31}
\item[Author]\AsIs{Moo, The cow who believes he's a dog!}
\item[Maintainer]\AsIs{Moo, The cow who believes he's a dog! }\email{moo@somewhere.com}\AsIs{}
\item[Description]\AsIs{More details about what the package does. See
<http://cran.r-project.org/doc/manuals/r-release/R-exts.html\#The-DESCRIPTION-file> for details
on how to write this part.}
\item[License]\AsIs{GPL (>= 2)}
\item[NeedsCompilation]\AsIs{no}
\end{description}
\Rdcontents{\R{} topics documented:}
\inputencoding{utf8}
\HeaderA{moo-package}{What the Package Does Using Title Case}{moo.Rdash.package}
\aliasA{moo}{moo-package}{moo}
\keyword{package}{moo-package}
%
\begin{Description}\relax
More details about what the package does. See
           <http://cran.r-project.org/doc/manuals/r-release/R-exts.html\#The-DESCRIPTION-file> for details
           on how to write this part.
\end{Description}
%
\begin{Section}{Package Content}

Index of help topics:
\begin{alltt}
hello                   A simple function doing little
moo-package             What the Package Does Using Title Case
\end{alltt}
\end{Section}
%
\begin{Section}{Maintainer}
Moo, The cow who believes he's a dog! <moo@somewhere.com>
\end{Section}
%
\begin{Author}\relax
Moo, The cow who believes he's a dog!
\end{Author}
\inputencoding{utf8}
\HeaderA{hello}{A simple function doing little}{hello}
%
\begin{Description}\relax
This function shows a standard text on the console. In a time-honoured
tradition, it defaults to displaying \emph{hello, world}.
\end{Description}
%
\begin{Examples}
\begin{ExampleCode}
  hello()
  hello("and goodbye")
\end{ExampleCode}
\end{Examples}
\printindex{}
\end{document}
